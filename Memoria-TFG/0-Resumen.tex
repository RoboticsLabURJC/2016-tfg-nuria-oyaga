\chapter*{Resumen}

En los últimos años, la investigación para conseguir que las máquinas perciban escenas y estímulos visuales con la robustez y rapidez que lo hacen los humanos, ha sido altamente desarrollada. En este aspecto, el campo de la Inteligencia Artificial, y en concreto el aprendizaje máquina como el aprendizaje profundo (\textit{Deep Learning}) con redes neuronales convolucionales, han experimentado un claro avance, permitiendo por ejemplo la identificación robusta de personas o la detección fiable de vehículos. Este tipo de redes precisa de un entrenamiento previo utilizando bases de datos con gran número de muestras, hecho que las diferencia de otro tipo de redes. Este trabajo analiza las redes neuronales convolucionales mediante el uso de la plataforma Caffe, que permite el entrenamiento de estas redes y facilita el trabajo de la implementación en aplicaciones propias.\\

Las redes neuronales en visión artificial pueden ser utilizadas para la resolución de dos problemas. Por un lado, clasificar el contenido de una imagen en una categoría de entre un conjunto de clases indicadas en el entrenamiento. Por otro lado, detectar estímulos dentro de la imagen, obteniendo como salida un conjunto de cajas delimitadoras que indiquen la presencia de estos estímulos, según el entrenamiento previo.\\

En este trabajo se ha construido un componente en Python que clasifica números manuscritos utilizando una red neuronal de Caffe. También se ha estudiado el efecto que entrenar con diferentes bases de datos tiene sobre las prestaciones del clasificador. Se ha utilizado la base de datos MNIST, compuesta por imágenes de dígitos manuscritos. A esta base de datos se le ha aplicado un preprocesamiento para lograr una red más robusta y numerosa. En concreto, se han considerado transformaciones de escalado, rotación, traslación y contaminación con ruido aditivo. Posteriormente, se ha aplicado un filtro de bordes de Sobel, para diseñar una red que premita resolver el problema independientemente de la procedencia de las imágenes (base de datos original, imágenes captadas con una cámara). Adicionalmente, tambien se ha explorado la detección de estímulos interesantes con redes neuronales de Caffe.