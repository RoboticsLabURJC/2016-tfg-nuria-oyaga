\chapter*{Resumen}

En los últimos años, la investigación para conseguir que las máquinas perciban escenas y estímulos visuales con la robustez y rapidez que lo hacen los humanos, ha sido altamente desarrollada. En este aspecto, el campo de la Inteligencia Artificial, y en concreto el aprendizaje profundo (\textit{Deep Learning}) con redes neuronales convolucionales, han experimentado un claro avance, permitiendo por ejemplo la identificación robusta de personas o la detección fiable de vehículos. En este trabajo se profundiza en el campo de las redes neuronales convolucionales mediante el uso de la plataforma Caffe, que permite el entrenamiento de estas redes y facilita el trabajo de la implementación en aplicaciones propias.\\

Las redes neuronales en visión artificial pueden ser utilizadas para la resolución de dos problemas. Por un lado, clasificar una imagen en un conjunto de clases que fueron indicadas en el entrenamiento.  Para este problema la salida será siempre una de las clases posibles, aunque se cometa un error. Por otro lado, detectar estímulos dentro de la imagen, obteniendo como salida una, ninguna o varias cajas delimitadoras, de objetos para cuya detección fue entrenada la red.\\

En este TFG se ha construido un componente en Python que clasifica números manuscritos utilizando una red
neuronal de Caffe. También se ha estudiado el efecto en el rendimiento del clasificador de entrenar con diferentes bases de datos a la red neuronal. Se ha utilizado el conjunto de datos MNIST en el entrenamiento, siendo modificado el contenido de las imágenes según los resultados obtenidos, pero sin modificar el número de muestras. Se ha aplicado al conjunto una mezcla de las transformaciones típicas, escalado, rotación, traslación y ruido, y posteriormente, un filtro de bordes de Sobel, consiguiendo una red lo suficientemente robusta para la aplicación.\\

También se ha explorado en este TFG el empleo de redes neuronales de Caffe para detectar estímulos interesantes. En este aspecto, Caffe emplea la rama SSD, cuyos resultados tras la elaboración del trabajo no han sido concluyentes, por lo que se deberá de continuar con este aspecto en un futuro.