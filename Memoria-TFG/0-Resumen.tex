\chapter*{Resumen}

En los últimos años, la investigación para conseguir que las máquinas consigan realizar tareas sencillas para el ser humano por su intuitividad, ha sido altamente desarrollada. En este aspecto, el campo de la Inteligencia Artificial, y en concreto el aprendizaje profundo con redes neuronales convolucionales, han experimentado un claro avance, permitiendo realizar tareas como la identificación de personas o la detección de vehículos. En este trabajo se profundiza en el campo de las redes neuronales convolucionales mediante el uso de la plataforma Caffe, una herramienta que permite el entrenamiento de estas redes y facilita el trabajo de la implementación en aplicaciones propias.\\

Las redes neuronales pueden ser utilizadas para la resolución de los dos problemas que se le plantea al ser humano en el campo de interés. Por un lado, es posible realizar la clasificación de unos determinados datos de entrada, por ejemplo una imagen, en un conjunto de clases que fueron indicadas en el entrenamiento.  Para este problema la salida será siempre una de las clases posibles, aunque se cometa un error. Por otro lado, se permite la detección, obteniendo como salida una, ninguna o varias cajas delimitadoras, en función de si se decide la presencia de los objetos, para los que fue entrenada la red, en los datos de entrada.\\

El problema de la clasificación tratado con Caffe se ha enfocado en elaborar un clasificador de dígitos en Python, cuya red empleada ha sido mejorada al comprobar distintos resultados. Se ha utilizado el conjunto de datos MNIST en el entrenamiento, siendo modificado el contenido de las imágenes según los resultados obtenidos, pero sin modificar el número de muestras. Se ha aplicado al conjunto una mezcla de las transformaciones típicas, escalado, rotación, traslación y ruido, y posteriormente, un filtro de bordes de Sobel, consiguiendo una red lo suficientemente robusta para la aplicación.\\

Finalmente, para la detección, Caffe emplea la rama SSD, cuyos resultados tras la elaboración del trabajo no han sido concluyentes, por lo que se deberá de continuar con este aspecto en un futuro.