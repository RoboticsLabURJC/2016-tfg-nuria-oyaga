\chapter{Conclusiones y líneas futuras}\label{cap.conclusiones}
En este capítulo se exponen las conclusiones alcanzadas con el desarrollo del trabajo, así como posibles líneas para continuar en el futuro.

\section{Conclusiones}
El desarrollo de este trabajo ha permitido establecer una serie de conclusiones en el ámbito de aprendizaje profundo mediante el empleo de redes neuronales entrenadas con Caffe. Estas conclusiones son expuestas a continuación, divididas según los sub-objetivos que se establecieron en la Sección~\ref{sec.objetivos}.
\begin{description}
	\item[Estudio de la plataforma Caffe.] \hfill 
	\vspace{5pt}
	\\
	La plataforma Caffe posee un funcionamiento muy adecuado para el ámbito de la visión artificial gracias a un entrenamiento rápido y sencillo. También posee un gran número de ejemplos que hacen más sencillo el trabajo con la misma y permite abarcar problemas de clasificación y detección.
	\item[Desarrollo de componente clasificador de dígitos.] \hfill 
	\vspace{5pt}
	\\
	El componente Python obtenido resulta bastante robusto gracias a la implementación de la mejor red obtenida tras realizar diferentes pruebas. Este componente permite la clasificación de dígitos mostrados a una cámara RGB, obteniendo buenos resultados independientemente de la procedencia de los mismos.\\
	\item[Desarrollo de un banco de pruebas.] \hfill 
	\vspace{5pt}
	\\
	Este banco de pruebas fue implementado gracias a un conjunto sencillo de hojas de cálculo que implementa las fórmulas necesarias. Este banco ha sido utilizado como herramienta para obtener las medidas de prestaciones necesarias para comparar las distintas redes.
	\item[Estudio y mejora de redes neuronales para la clasificación de dígitos.] \hfill 
	\vspace{5pt}
	\\
	Este sub-objetivo ha sido el bloque central del trabajo, por lo que el número de conclusiones alcanzadas sobre el mismo es superior a los demás. Tras alcanzar el sub-objetivo se ha concluido que:
	\begin{itemize}
		\item El entrenamiento con imágenes de borde amplía el número aciertos en la clasificación, pues independiza la representación del contenido de la imagen de los niveles de intensidad.
		\item Al realizar una serie de  transformaciones en las imágenes y considerarlas en el entrenamiento de la red, las prestaciones de la red mejoran, permitiendo obtener una red más robusta.
		\item Las prestaciones obtenidas al incluir un alto número de imágenes transformadas frente a la que se obtiene disminuyendo ese número de imágenes es muy similar, por lo que no es necesario incluir un gran número de estas imágenes en el entrenamiento.
		\item Sobre la inclusión de la imagen limpia en el entrenamiento, siendo ésta la imagen sin ninguna transformación, se obtienen resultados que avalan que no es necesario incluirla para obtener mejores prestaciones.
	\end{itemize}

	\item[Primera aproximación a la detección con Caffe.] \hfill 
	\vspace{5pt}
	\\
	Los resultados obtenidos en este ámbito hacen ver que el problema de detección es muy amplio. Según la base de datos empleada en el entrenamiento, la estructura de la red, y el estímulo que se desee detectar, los resultados de la detección y las prestaciones de la red varían. La plataforma proporciona varias herramientas que permiten implementar redes neuronales para la detección \acrshort{ssd}.
\end{description}

\section{Líneas futuras}
Para seguir con la investigación abordada en este trabajo se pueden seguir varias vías que permitan obtener resultados interesantes en el campo del aprendizaje profundo.
\begin{itemize}
	\item En la tarea de clasificación, es posible extender lo aprendido para un ejemplo sencillo, la clasificación de dígitos con redes neuronales gracias al conjunto de datos \acrshort{mnist}, a un ejemplo más complejo como puede ser la clasificación de signos o de señales de tráfico.
	\item Los resultados obtenidos en el campo de la detección dejan ver una gran línea de investigación que permita obtener una red robusta para la detección de estímulos interesantes.
	\item Es posible realizar un estudio de los mismos problemas tratados en este trabajo con herramientas diferentes, como TensorFlow. De esta manera, es posible contrastar distintas redes para la solución de un mismo problema, permitiendo la elección de aquélla más adecuada a cada problema.
	\item Es posible realizar el estudio de redes desarrolladas con Caffe para estímulos no visuales, que permitan solucionar problemas de distinta naturaleza.
	\item En relación al banco de pruebas, es posible tratar de integrar las redes desarrolladas con Caffe en un banco automático, desarrollado por David Pascual\footnote{https://github.com/RoboticsURJC-students/2016-tfg-david-pascual}, que agilice la obtención de parámetros de evaluación.
\end{itemize}

Finalmente, la aplicación más completa consiste en la resolución de ambos problemas. De esta manera, por ejemplo, sería posible la elaboración de una aplicación que permitiese a un choche circular de forma autónoma, detectando los estímulos que recibe de la carretera y clasificándolos para actuar en consecuencia. En esta línea de investigación se encuentra la organización de Udacity, que proporciona una base de datos con horas de conducción para el desarrollo de una red que permita implementar la conducción autónoma con aprendizaje profundo.\\

Los anteriores son solo algunos ejemplos de los problemas que se pueden tratar de abordar con esta tecnología. La cantidad de tareas que se pueden realizar es tan amplia como el número de problemas que se le pueden presentar a un ser humano en su día a día, pues la finalidad de estas aplicaciones no es otra que facilitar la vida diaria de las personas.